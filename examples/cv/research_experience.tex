%-------------------------------------------------------------------------------
%	SECTION TITLE
%-------------------------------------------------------------------------------
\cvsection{Research Experience}
\label{sec:exp}
%\vspace{-5mm}
%-------------------------------------------------------------------------------
%	CONTENT
%-------------------------------------------------------------------------------

%\descriptionstyle{	
%	Grants: Authorship of \textbf{successful} research proposals with funding totaling \textbf{$>$1.7M USD} \\
%	Field experience: Organization/participation of/in aerial-robotic field-campaigns to the Arctic, Antarctic, Amazon, and Swiss/Italian Alps
%	}
\vspace{-8pt}
\begin{cventries}
%\begin{multicols}{2}
%
%---------------------------------------------------------
\cvexpentry
  	{Autonomous Systems Lab (ASL), ETH Z\"{u}rich} % Institute
  	{Post-Doctoral Researcher} % Position
    {} % Subject
    {Since 2020} % Date(s)
    {
      \begin{cvitems} % Description(s) bullet points
      	\item Supervise and coordinate PhD and Masters student research activities related to aerodynamic modeling, system identification, estimation, control, and planning for fixed-wing and hybrid, tilt-wing, VTOL UAVs, some recent results including:
		\begin{itemize}
			\item automatic tilt-wing UAV control, stabilized deep stalled flight, and development and analysis of span and chord-wise wing-fitted pressure sensors for in-flight airflow characterization
		\end{itemize}		      		
      	\item Project lead for an Armasuisse S+T funded project on autonomous, high-speed, aerial, vision-based payload recovery using fixed-wing UAVs.
      	\item Project lead for a Swiss Polar Institute (SPI) funded project on autonomous, precision sensor placement and recovery on remote glaciers using a long-range tilt-wing UAV.
      	\item Project lead for ETH Foundation project ``AvalMapper'', on developing an autonomous aerial detection and mapping system for high-alpine avalanches utilizing machine learned classification methods and informative path planning for reliably reconstructable snow-depth maps.
      \end{cvitems}
    } % Description
    {} % website
    {true}
    {}
    %\vspace{-12pt}
%
%---------------------------------------------------------
\cvexpentry
  	{Autonomous Systems Lab (ASL), ETH Z\"{u}rich} % Institute
  	{PhD Research Assistant} % Position
    {} % Subject
    {2014 - 2020} % Date(s)
    {
      \begin{cvitems} % Description(s) bullet points
      	\item Core researcher on EU search-and-rescue robotics projects \emph{SHERPA} and \emph{ICARUS}, organizing multiple university and industry partners in collaborative multi-robotic field demonstrations. \weblink{https://www.euronews.com/2016/05/23/dealing-with-danger-busy-geniuses-and-watchful-robots}{https://www.euronews.com/2016/05/23/dealing-with-danger-busy-geniuses-and-watchful-robots}
		\item Interfaced with customers and industry partners within the ESA precision-farming project \emph{SOLAR3} to deliver a reliable automatic, multi-hour endurance, solar-powered surveying drone solution to non-expert end-users in Switzerland and Ukraine.      	
      	\item Developed and deployed robust, wind-aware estimation, guidance, and control algorithms for multiple classes of UAVs in extreme wind conditions.
      	\item Designed and deployed Nonlinear Model Predictive Control (NMPC) schemes for/on fixed-wing UAVs including objectives for agressive 3D path following, actuator fault tolerance, stall prevention, and vision-based terrain feedback.
      	\item Developed and utilized a semi-automated system identification pipeline for fixed-wing and hybrid UAVs using iterated ekf based flight path reconstruction and nonlinear grey-box optimization for identifying full envelope parameterized aircraft models from flight data.
      	\item Conducted performance optimization and developed automatic take-off, landing, and cruise control design for the \emph{AtlantikSolar UAV}, resulting in an \textbf{81.5 hour endurance world record} solar-powered flight for aircraft $<$50kg \weblink{http://www.atlantiksolar.ethz.ch/index.html\%3Fp=670.html}{http://www.atlantiksolar.ethz.ch/index.html\%3Fp=670.html} and 26 hour, fully autonomous, search-and-rescue payload equipped flight \weblink{http://www.atlantiksolar.ethz.ch/index.html\%3Fp=931.html}{http://www.atlantiksolar.ethz.ch/index.html\%3Fp=931.html}
      \end{cvitems}
    } % Description
    {} % website
    {true}
    {}
%   
%---------------------------------------------------------
\cvexpentry
  	{Center for Remote Sensing of Ice Sheets (CReSIS), University of Kansas} % Institute
  	{Masters Research Assistant} % Position
    {} % Subject
    {2012 - 2014} % Date(s)
    {
      \begin{cvitems} % Description(s) bullet points
      	\item Conducted research on control and planning for fixed-wing UAVs including multi-agent avoidance and formation strategies.
      	\item Contributed to the design, integration, and \textbf{Antarctic deployment} of a polar-conditioned fixed-wing UAV with integrated dual-frequency ground-penetrating radar.
      \end{cvitems}
    } % Description
    {} % website
    {true}
    {}
\end{cventries}
