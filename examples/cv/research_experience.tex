%-------------------------------------------------------------------------------
%	SECTION TITLE
%-------------------------------------------------------------------------------
\cvsection{Research Experience}
\label{sec:exp}
%\vspace{-5mm}
%-------------------------------------------------------------------------------
%	CONTENT
%-------------------------------------------------------------------------------

%\descriptionstyle{	
%	Grants: Authorship of \textbf{successful} research proposals with funding totaling \textbf{$>$1.7M USD} \\
%	Field experience: Organization/participation of/in aerial-robotic field-campaigns to the Arctic, Antarctic, Amazon, and Swiss/Italian Alps
%	}
\vspace{-8pt}
\begin{cventries}
%\begin{multicols}{2}
%
%---------------------------------------------------------
\cvexpentry
  	{Autonomous Systems Lab (ASL), ETH Z\"{u}rich} % Institute
  	{Post-Doctoral Researcher} % Position
    {} % Subject
    {Since 10/2020} % Date(s)
    {
      \begin{cvitems} % Description(s) bullet points
      	\item Supervise and coordinate PhD and Masters student research activities related to fixed-wing UAVs, recent results including:
		\begin{itemize}
			\item automatic tilt-wing UAV control, stabilized deep stalled flight, span/chord-wise wing-fitted pressure sensor arrays for in-flight airflow measurement, design and wind tunnel characterization of a winged omni-directional UAV
		\end{itemize}
      	\item Project lead for on-going funded projects:
      	\begin{itemize}
      		\item high-speed vision-based payload recovery using fixed-wing UAVs
      		\item precision sensor placement and recovery on remote glaciers using a long-range tilt-wing UAV
      		\item ``AvalMapper'', developing an autonomous aerial detection and mapping system for high-alpine avalanches utilizing machine learned classification methods and informative path planning for reliably reconstructable snow-depth maps
		\end{itemize}      	 
      \end{cvitems}
    } % Description
    {} % website
    {true}
    {}
    %\vspace{-12pt}
%
%---------------------------------------------------------
\cvexpentry
  	{Autonomous Systems Lab (ASL), ETH Z\"{u}rich} % Institute
  	{PhD Research Assistant} % Position
    {} % Subject
    {2014 - 2020} % Date(s)
    {
      \begin{cvitems} % Description(s) bullet points
%      	\item Core researcher on EU search-and-rescue robotics projects \emph{SHERPA} and \emph{ICARUS}, organizing multiple university and industry partners in collaborative multi-robotic field demonstrations.
%		\item Interfaced with customers and industry partners within the ESA precision-farming project \emph{SOLAR3} to deliver a reliable automatic, multi-hour endurance, surveying drone solution to non-expert end-users in Switzerland and Ukraine.      	
%      	\item Developed and deployed:
%      	\begin{itemize}
%      		\item efficient wind-aware guidance and control algorithms for multiple classes of UAVs in extreme weather conditions
%      		\item Nonlinear Model Predictive Control (NMPC) algorithms for/on fixed-wing UAVs including fault tolerance, stall prevention, and vision-based terrain feedback
%      		\item a semi-automated system identification pipeline for fixed-wing UAVs from flight data to full envelope simulation model
%      	\end{itemize}
%      	\item Conducted performance optimization and developed automatic take-off, landing, and cruise control design for the \emph{AtlantikSolar UAV}, resulting in an \textbf{81.5 hour endurance world record} solar-powered flight for aircraft $<$50kg and 26 hour, fully autonomous, search-and-rescue payload equipped flight.
		\item Core researcher on EU search-and-rescue robotics projects \emph{SHERPA} and \emph{ICARUS}, organizing multiple university and industry partners in collaborative multi-robotic field demonstrations.
		\item Interfaced with customers and industry partners within the ESA precision-farming project \emph{SOLAR3} to deliver a reliable automatic, multi-hour endurance, solar-powered surveying drone solution to non-expert end-users in Switzerland and Ukraine.      	
      	\item Developed and deployed:
		\begin{itemize}
			\item robust, wind-aware estimation, guidance, and control algorithms for UAVs in extreme wind conditions
      		\item nonlinear model predictive control (NMPC) schemes for fixed-wing UAVs including objectives for aggressive 3D path following, actuator fault tolerance, stall prevention, and vision-based terrain feedback
      		\item a semi-automated system identification pipeline for fixed-wing UAVs (iterated EKF, nonlinear parameter optimization)
		\end{itemize}		      	
      	\item Conducted performance optimization and developed automatic take-off, landing, and cruise control design for the \emph{AtlantikSolar UAV}, resulting in an \textbf{81.5 hour endurance world record} solar-powered flight for aircraft $<$50kg \weblink{https://youtu.be/8m4\_NpTQn0E}{https://youtu.be/8m4\_NpTQn0E} and 26 hour, fully autonomous, search-and-rescue payload equipped flight \weblink{https://youtu.be/8m76Mx9m2nM}{https://youtu.be/8m76Mx9m2nM}
      \end{cvitems}
    } % Description
    {} % website
    {true}
    {}
%   
%---------------------------------------------------------
\cvexpentry
  	{Center for Remote Sensing of Ice Sheets (CReSIS), University of Kansas} % Institute
  	{Masters Research Assistant} % Position
    {} % Subject
    {2012 - 2014} % Date(s)
    {
      \begin{cvitems} % Description(s) bullet points
      	\item Conducted research on control and planning for fixed-wing UAVs including multi-agent avoidance and formation strategies.
      	\item Contributed to the design, integration, and \textbf{Antarctic deployment} of a polar-conditioned fixed-wing UAV with integrated dual-frequency ground-penetrating radar.
      \end{cvitems}
    } % Description
    {} % website
    {true}
    {}
\end{cventries}
