%-------------------------------------------------------------------------------
%	SECTION TITLE
%-------------------------------------------------------------------------------
\cvsection{Research Experience}{}
\label{sec:exp}
%\vspace{-5mm}
%-------------------------------------------------------------------------------
%	CONTENT
%-------------------------------------------------------------------------------

%\descriptionstyle{	
%	Grants: Authorship of \textbf{successful} research proposals with funding totaling \textbf{$>$1.7M USD} \\
%	Field experience: Organization/participation of/in aerial-robotic field-campaigns to the Arctic, Antarctic, Amazon, and Swiss/Italian Alps
%	}
\vspace{-8pt}
\begin{cventries}
%\begin{multicols}{2}
%
%---------------------------------------------------------
\cvexpentry
  	{Autonomous Systems Lab (ASL), ETH Z\"{u}rich} % Institute
  	{Post-Doctoral Researcher} % Position
    {} % Subject
    {Since 6/2020} % Date(s)
    {
      \begin{cvitems} % Description(s) bullet points
      	\item Supervise and coordinate PhD and Masters student research activities related to measurement, aerodynamic modeling, system identification, and control of fixed-wing and hybrid, tilt-wing, VTOL UAVs, results including:
      	\begin{itemize}\setlength{\parskip}{0pt}
      		\item automatic tilt-wing control -- video: \weblink{https://youtu.be/pSXEnHUY2\_4}{https://youtu.be/pSXEnHUY2\_4}
      		\item stabilized deep stalled flight -- video: \weblink{https://drive.google.com/file/d/1JpexWpThE5TOrnXN1Og9uz9aQ5ysgh-m/view?usp=sharing}{https://drive.google.com/file/d/1JpexWpThE5TOrnXN1Og9uz9aQ5ysgh-m/view?usp=sharing}
    		\end{itemize}
      	\item Lead a team of PhD and Masters students on an (ongoing) project for autonomous, high-speed, aerial, vision-based payload recovery.
      \end{cvitems}
    } % Description
    {} % website
    {true}
    {}
    %\vspace{-12pt}
%
%---------------------------------------------------------
\cvexpentry
  	{Autonomous Systems Lab (ASL), ETH Z\"{u}rich} % Institute
  	{PhD Research Assistant} % Position
    {} % Subject
    {2014 - 2020} % Date(s)
    {
      \begin{cvitems} % Description(s) bullet points
      	\item Core researcher on EU search-and-rescue robotics projects \emph{SHERPA} and \emph{ICARUS}, organizing multiple university and industry partners in collaborative multi-robotic field demonstrations.
		\item Interfaced with customers and industry partners within the ESA precision-farming project \emph{SOLAR3} to deliver a reliable automatic, multi-hour endurance, surveying drone solution to non-expert end-users in Switzerland and Ukraine.      	
      	\item Developed and deployed:
      	\begin{itemize}
      		\item efficient wind-aware guidance and control algorithms for multiple classes of UAVs in extreme weather conditions
      		\item Nonlinear Model Predictive Control (NMPC) algorithms for/on fixed-wing UAVs including fault tolerance, stall prevention, and vision-based terrain feedback
      		\item a semi-automated system identification pipeline for fixed-wing UAVs from flight data to full envelope simulation model
      	\end{itemize}
      	\item Conducted performance optimization and developed automatic take-off, landing, and cruise control design for the \emph{AtlantikSolar UAV}, resulting in an \textbf{81.5 hour endurance world record} solar-powered flight for aircraft $<$50kg and 26 hour, fully autonomous, search-and-rescue payload equipped flight.
      \end{cvitems}
    } % Description
    {} % website
    {true}
    {}
%   
%---------------------------------------------------------
\cvexpentry
  	{Center for Remote Sensing of Ice Sheets (CReSIS), University of Kansas} % Institute
  	{Masters Research Assistant} % Position
    {} % Subject
    {2012 - 2014} % Date(s)
    {
      \begin{cvitems} % Description(s) bullet points
      	\item Conducted research on control and planning for fixed-wing UAVs including multi-agent avoidance and formation strategies.
      	\item Contributed to the design, integration, and \textbf{Antarctic deployment} of a polar-conditioned fixed-wing UAV with integrated dual-frequency ground-penetrating radar.
      \end{cvitems}
    } % Description
    {} % website
    {true}
    {}
\end{cventries}
